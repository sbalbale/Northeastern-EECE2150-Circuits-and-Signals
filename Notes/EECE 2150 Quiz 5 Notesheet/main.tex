\documentclass[9pt]{article}
\usepackage[margin=0.4in]{geometry}
\usepackage{amsmath, amssymb}
\usepackage{multicol}
\usepackage{graphicx}
\usepackage{circuitikz}

\setlength{\columnsep}{0.8cm}
\setlength{\parindent}{0in}

\title{
    \vspace{-2em}
    \large EECE 2150: Quiz 5 Notesheet \hfill Sean Balbale \hfill November 6, 2024
    \vspace{-4em}
}
\date{}

\begin{document}
\maketitle

\begin{multicols*}{2}

% Basic Circuit Elements
\section{Basic Circuit Elements}

\subsection{Resistors}
\textbf{Resistor} resists the flow of charge. The resistance \(R\) is a function of length, area, and resistivity:
\[
R = \frac{\rho \ell}{A} \quad \text{or} \quad R = \frac{\ell}{\sigma A}
\]
\begin{itemize}\itemsep0pt
    \item \( \rho \): Resistivity
    \item \( \ell \): Length
    \item \( A \): Area
\end{itemize}

\textbf{Ohm's Law}: Voltage, current, and resistance are related:
\[ V = IR \quad \text{or} \quad I = \frac{V}{R} \]

\subsection{Power Dissipation and Passive Sign Convention}
\textbf{Power Dissipation}:
\[ P = IV = I^2 R = \frac{V^2}{R} \]
- In passive sign convention, if current enters positive terminal, element absorbs power: \[ P = VI \]
- If current enters negative terminal, element delivers power: \[ P = -VI \]

\subsection{Conductance}
Conductance is the reciprocal of resistance:
\[ G = \frac{1}{R} \quad \text{in Siemens (S)} \]

\subsection{Ideal Conductors}
\[ R = 0, \quad \sigma \to \infty \]
No voltage drop across an ideal conductor.

--- 

% Kirchhoff’s Laws
\section{Kirchhoff’s Laws}

\subsection{Kirchhoff's Current Law (KCL)}
The sum of currents entering and leaving a node is zero:
\[ \sum I_{\text{in}} = \sum I_{\text{out}} \]

\subsection{Kirchhoff's Voltage Law (KVL)}
The sum of voltage drops in a closed loop equals the sum of voltage rises:
\[ \sum V_{\text{drops}} = \sum V_{\text{rises}} \]

--- 

% Resistors in Series and Parallel
\section{Resistors in Series and Parallel}

\subsection{Series Resistors}
Resistors in series carry the same current:
\[ R_{\text{eq}} = R_1 + R_2 + \cdots + R_n \]
\textbf{Voltage Division}:
\[ V_k = V_s \frac{R_k}{R_{\text{eq}}} \]

\subsection{Parallel Resistors}
Resistors in parallel share the same voltage:
\[ \frac{1}{R_{\text{eq}}} = \frac{1}{R_1} + \frac{1}{R_2} + \cdots + \frac{1}{R_n} \]
\textbf{Current Division}:
\[ i_k = I_s \frac{R_{\text{eq}}}{R_k} \]

--- 

% Nodal Analysis (Node Voltage Method)
\section{Nodal Analysis (Node Voltage Method)}

\subsection{Steps for Nodal Analysis}
\begin{enumerate}\itemsep0pt
    \item Identify essential nodes.
    \item Choose a reference node (ground).
    \item Write KCL equations at each essential node using node voltages.
    \item Solve the system of equations for unknown node voltages.
\end{enumerate}

--- 

% Mesh Analysis (Mesh Current Method)
\section{Mesh Analysis (Mesh Current Method)}

\subsection{Steps for Mesh Analysis}
\begin{enumerate}\itemsep0pt
    \item Identify meshes (loops without other loops inside).
    \item Assign mesh currents.
    \item Apply KVL in each mesh to write voltage equations.
    \item Solve the system of equations for mesh currents.
\end{enumerate}

--- 

% Thevenin and Norton Equivalent Circuits
\section{Thevenin and Norton Equivalent Circuits}

\subsection{Thevenin's Theorem}
Any linear circuit can be reduced to a single voltage source \( V_{\text{Th}} \) in series with \( R_{\text{Th}} \).
\[ V_L=V_{Th}\frac{R_L}{R_{Th}+R_L} \]

\subsection{Norton's Theorem}
Any linear circuit can be reduced to a single current source \( I_N \) in parallel with \( R_N \).
\[ I_N = \frac{V_{\text{Th}}}{R_{\text{Th}}}, \quad R_N = R_{\text{Th}} \]

--- 

% Source Transformations
\section{Source Transformations}

\subsection{Voltage to Current Source Transformation}
A voltage source \( V_s \) in series with \( R \) can be transformed into a current source:
\[ I_s = \frac{V_s}{R}, \quad \text{in parallel with } R \]

\subsection{Current to Voltage Source Transformation}
A current source \( I_s \) in parallel with \( R \) can be transformed into a voltage source:
\[ V_s = I_s R, \quad \text{in series with } R \]

--- 

% Delta-Y (Δ-Y) Conversion
\section{Delta-Y ($\Delta$-Y) Conversion}

\subsection{Delta to Y Conversion}
For a delta network with resistors \(R_a\), \(R_b\), and \(R_c\), the equivalent Y-resistances are:
\[ R_1 = \frac{R_b R_c}{R_a + R_b + R_c}, \quad R_2 = \frac{R_a R_c}{R_a + R_b + R_c} \]
\[ R_3 = \frac{R_a R_b}{R_a + R_b + R_c} \]

\subsection{Y to Delta Conversion}
For a Y-network with resistors \(R_1\), \(R_2\), and \(R_3\), the equivalent delta-resistances are:
\[ R_a = \frac{R_1 R_2 + R_2 R_3 + R_3 R_1}{R_3}, \quad R_b = \frac{R_1 R_2 + R_2 R_3 + R_3 R_1}{R_1} \]
\[ R_c = \frac{R_1 R_2 + R_2 R_3 + R_3 R_1}{R_2} \]

--- 

% Operational Amplifiers
\section{Operational Amplifiers (Op-Amps)}

\subsection{Introduction to Op-Amps}
\textbf{Operational Amplifier (Op-Amp)} is a high-gain IC used in various mathematical and amplification operations:
\begin{itemize}\itemsep0pt
    \item Ideal characteristics: infinite input impedance, zero output impedance, infinite open-loop gain, and bandwidth.
    \item Common applications: integrator, differentiator, summing amplifier, buffer amplifier, difference amplifier.
\end{itemize}

\subsection{Ideal Op-Amp Model}
\begin{itemize}\itemsep0pt
    \item Infinite input impedance \(R_{in} = \infty\).
    \item Zero output impedance \(R_{out} = 0\).
    \item Infinite open-loop gain \(A_{vo} \to \infty\).
    \item Differential input voltage \(V_d = V^+ - V^-\), ideally \(V_d \approx 0\).
\end{itemize}

\subsection{Basic Op-Amp Configurations}

\subsubsection{Inverting Amplifier}
Gain \(A_v = -\frac{R_f}{R_{in}}\)
\[
V_{out} = -V_{in} \cdot \frac{R_f}{R_{in}}
\]

\subsubsection{Non-Inverting Amplifier}
Gain \(A_v = 1 + \frac{R_f}{R_1}\)
\[
V_{out} = V_{in} \cdot \left(1 + \frac{R_f}{R_1}\right)
\]

\subsubsection{Voltage Follower (Buffer)}
Gain \(A_v = 1\), used for isolation
\[
V_{out} = V_{in}
\]

\subsection{Real-World Op-Amp Limitations}

\subsubsection{Bandwidth and Slew Rate}
\textbf{Bandwidth:} Gain-Bandwidth Product (GBP): \(A_v \times f_{BW} = \text{constant}\)
\textbf{Slew Rate:} Max rate of change of \( V_{out} \):
\[
\text{Slew Rate} = \frac{\Delta V_{out}}{\Delta t}
\]

\subsubsection{Input Offset Voltage and Bias Current}
\textbf{Input Offset Voltage} \(V_{OS}\): Small voltage difference to zero \(V_{out}\).
\textbf{Input Bias Current:} Average current into input terminals.

\subsection{Op-Amp Analysis Techniques}
\begin{itemize}
    \item Assume ideal conditions (\(V^+ = V^-\), \(I^+ = I^- = 0\)).
    \item Apply KCL at nodes, considering virtual ground in inverting amplifiers.
\end{itemize}

\subsection{Feedback in Op-Amps}
\textbf{Types of Feedback:}
\begin{itemize}\itemsep0pt
    \item Negative Feedback: Stabilizes gain.
    \item Positive Feedback: Used in oscillators.
\end{itemize}
Closed-Loop Gain:
\[
A_{CL} = \frac{A_{vo}}{1 + A_{vo} \cdot \beta}
\]

\end{multicols*}
\end{document}
