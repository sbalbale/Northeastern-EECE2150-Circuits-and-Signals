\documentclass[9pt]{article}
\usepackage[margin=0.4in]{geometry}
\usepackage{amsmath, amssymb}
\usepackage{multicol}
\usepackage{graphicx}
\usepackage{circuitikz}

\setlength{\columnsep}{0.8cm}
\setlength{\parindent}{0in}

\title{
    \vspace{-2em}
    \large EECE 2150: Quiz 6 Notesheet \hfill Sean Balbale \hfill November 20, 2024
    \vspace{-4em}
}
\date{}

\begin{document}
\maketitle

\begin{multicols}{2}

% Capacitors and Inductors
\section{Capacitors and Inductors}

\subsection{Capacitors}
A capacitor stores energy in the electric field between its plates.

\textbf{Key Relationships}:
\[
C = \frac{Q}{V}, \quad i(t) = C \frac{dv(t)}{dt}, \quad v(t) - v(t_0) = \frac{1}{C} \int_{t_0}^t i(\tau) d\tau
\]
\[
w(t) = \frac{1}{2} C v^2(t), \quad P(t) = i(t)v(t)
\]

\textbf{Behavior}:
\begin{itemize}\itemsep0pt
    \item Acts as an open circuit in steady-state DC.
    \item Voltage cannot change instantaneously (\(i \to \infty\) for rapid voltage changes).
    \item Frequency selective: responds differently to fast and slow voltage changes.
\end{itemize}

\textbf{Capacitor Combinations}:
\begin{itemize}\itemsep0pt
    \item Series: \( \frac{1}{C_{\text{eq}}} = \frac{1}{C_1} + \frac{1}{C_2} + \cdots \)
    \item Parallel: \( C_{\text{eq}} = C_1 + C_2 + \cdots \)
\end{itemize}

---

\subsection{Inductors}
An inductor stores energy in its magnetic field.

\textbf{Key Relationships}:
\[
v(t) = L \frac{di(t)}{dt}, \quad i(t) - i(t_0) = \frac{1}{L} \int_{t_0}^t v(\tau) d\tau
\]
\[
w(t) = \frac{1}{2} L i^2(t), \quad P(t) = v(t)i(t)
\]

\textbf{Behavior}:
\begin{itemize}\itemsep0pt
    \item Acts as a short circuit in steady-state DC.
    \item Current cannot change instantaneously (\(v \to \infty\) for rapid current changes).
    \item Frequency selective: opposes changes in current.
\end{itemize}

\textbf{Inductor Combinations}:
\begin{itemize}\itemsep0pt
    \item Series: \( L_{\text{eq}} = L_1 + L_2 + \cdots \)
    \item Parallel: \( \frac{1}{L_{\text{eq}}} = \frac{1}{L_1} + \frac{1}{L_2} + \cdots \)
\end{itemize}

---

% RC and RL Circuits
\section{RC and RL Circuits}

\subsection{RC Circuits}
\textbf{Capacitor Properties}:
\[
i(t) = C \frac{dv(t)}{dt}, \quad w(t) = \frac{1}{2} C v^2(t)
\]
- Capacitor acts as an open circuit in steady-state DC.

\textbf{Transient Analysis}:
\[
v(t) = V_f + (V_i - V_f)e^{-t/\tau}, \quad \tau = RC
\]
\[
i(t) = \frac{V_s}{R}e^{-t/RC}
\]

\subsection{RL Circuits}
\textbf{Inductor Properties}:
\[
v(t) = L \frac{di(t)}{dt}, \quad w(t) = \frac{1}{2} L i^2(t)
\]
- Inductor acts as a short circuit in steady-state DC.

\textbf{Transient Analysis}:
\[
i(t) = I_f + (I_i - I_f)e^{-t/\tau}, \quad \tau = \frac{L}{R}
\]
\[
v(t) = L \frac{di(t)}{dt}
\]

---

% Kirchhoff’s Laws
\section{Kirchhoff’s Laws}

\subsection{Kirchhoff's Current Law (KCL)}
The sum of currents entering and leaving a node is zero:
\[
\sum I_{\text{in}} = \sum I_{\text{out}}
\]

\subsection{Kirchhoff's Voltage Law (KVL)}
The sum of voltage drops in a closed loop equals the sum of voltage rises:
\[
\sum V_{\text{drops}} = \sum V_{\text{rises}}
\]

---

% Nodal Analysis
\section{Nodal Analysis}

\subsection{Steps for Nodal Analysis}
\begin{enumerate}\itemsep0pt
    \item Identify essential nodes.
    \item Choose a reference node (ground).
    \item Apply KCL at each node using node voltages.
    \item Solve the system of equations for unknown voltages.
\end{enumerate}

Example for RC Circuit:
\[
\frac{v_1 - v_2}{R} + C\frac{dv_2}{dt} = 0
\]

---

% Mesh Analysis
\section{Mesh Analysis}

\subsection{Steps for Mesh Analysis}
\begin{enumerate}\itemsep0pt
    \item Identify meshes (loops without other loops inside).
    \item Assign mesh currents.
    \item Apply KVL in each mesh.
    \item Solve for mesh currents.
\end{enumerate}

---

% Transient Responses
\section{Transient Responses}

\subsection{General Solution}
The total response is the sum of the natural and forced responses:
\[
y(t) = y_h(t) + y_p(t)
\]
- \(y_h(t)\): Homogeneous solution, exponential decay:
\[
y_h(t) = Ke^{-t/\tau}
\]
- \(y_p(t)\): Particular solution, steady-state response.

\subsection{Time Constant}
- RC Circuits: \( \tau = R_{eq}C \)
- RL Circuits: \( \tau = \frac{L}{R_{eq}} \)

---

% Example: Step Response
\section{Example: Step Response}

\subsection{RC Circuit}
Voltage across capacitor:
\[
v_C(t) = V(1 - e^{-t/RC})
\]
Current through resistor:
\[
i_R(t) = \frac{V}{R}e^{-t/RC}
\]

\subsection{RL Circuit}
Current through inductor:
\[
i_L(t) = I(1 - e^{-tR/L})
\]
Voltage across resistor:
\[
v_R(t) = IR(1 - e^{-tR/L})
\]

\end{multicols}
\end{document}
