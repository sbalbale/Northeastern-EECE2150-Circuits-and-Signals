\documentclass[10pt]{article}
\usepackage[margin=0.5in]{geometry}
\usepackage{amsmath, amssymb}
\usepackage{multicol}
\usepackage{graphicx}
\usepackage{circuitikz}

\setlength{\columnsep}{1.5cm} % Set the column separation to a larger value
\setlength{\voffset}{-0.25in}
\setlength{\parindent}{0in}

\title{
    \raggedright
    \large EECE 2150: Quiz 3 Notesheet \hfill Sean Balbale \hfill October 2024
    \vspace{-4em}
}
\date{}

\begin{document}
\maketitle

\begin{multicols}{2}

% SECTION 1: Basic Circuit Elements
\section*{1. Basic Circuit Elements}

\subsection*{1.1 Resistors}
\textbf{Resistor} resists the flow of charge. The resistance \(R\) is a function of length, area, and resistivity:
\[
R = \frac{\rho \ell}{A} \quad \text{or} \quad R = \frac{\ell}{\sigma A}
\]
\begin{itemize}\itemsep0pt
    \item \( \rho \): Resistivity
    \item \( \ell \): Length
    \item \( A \): Area
\end{itemize}

\textbf{Ohm's Law}: Voltage, current, and resistance are related:
\[
V = IR \quad \text{or} \quad I = \frac{V}{R}
\]

\subsection*{1.2 Power Dissipation and Passive Sign Convention}
\textbf{Power Dissipation}:
\[
P = IV = I^2 R = \frac{V^2}{R}
\]
- In the passive sign convention, if the current enters the positive terminal of an element, the element is absorbing power:
\[
P = VI
\]
- If the current enters the negative terminal, the element is delivering power:
\[
P = -VI
\]

\subsection*{1.3 Conductance}
Conductance is the reciprocal of resistance:
\[
G = \frac{1}{R} \quad \text{in Siemens (S)}
\]

\subsection*{1.4 Ideal Conductors}
\[ R = 0, \quad \sigma \to \infty \]
No voltage drop across an ideal conductor.

---

% SECTION 2: Kirchhoff’s Laws
\section*{2. Kirchhoff’s Laws}

\subsection*{2.1 Kirchhoff's Current Law (KCL)}
The sum of currents entering and leaving a node is zero:
\[
\sum I_{\text{in}} = \sum I_{\text{out}}
\]
\textit{Example}: If \( 2\,A \), \( 3\,A \), and \( 5\,A \) enter a node:
\[
i_x = 10\,A
\]

\subsection*{2.2 Kirchhoff's Voltage Law (KVL)}
The sum of voltage drops in a closed loop equals the sum of voltage rises:
\[
\sum V_{\text{drops}} = \sum V_{\text{rises}}
\]
\textit{Example}: In a loop with voltage drops of \(10\,V\), \(6\,V\), and \(4\,V\), and supply \(V_s = 24\,V\):
\[
V_x = 4\,V
\]

---

% SECTION 3: Resistors in Series and Parallel
\section*{3. Resistors in Series and Parallel}

\subsection*{3.1 Series Resistors}
Resistors in series carry the same current:
\[
R_{\text{eq}} = R_1 + R_2 + \cdots + R_n
\]
\textbf{Voltage Division}:
\[
V_k = V_s \frac{R_k}{R_{\text{eq}}}
\]
\textit{Example}: For two resistors \(R_1 = 10\,\Omega\), \(R_2 = 5\,\Omega\), and source \(V_s = 30\,V\):
\[
V_1 = 20\,V, \quad V_2 = 10\,V
\]

\subsection*{3.2 Parallel Resistors}
Resistors in parallel share the same voltage:
\[
\frac{1}{R_{\text{eq}}} = \frac{1}{R_1} + \frac{1}{R_2} + \cdots + \frac{1}{R_n}
\]
\textbf{Current Division}:
\[
i_k = I_s \frac{R_{\text{eq}}}{R_k}
\]
\textit{Example}: For two resistors \(R_1 = 6\,\Omega\), \(R_2 = 3\,\Omega\), and \(I_s = 6\,A\):
\[
R_{\text{eq}} = 2\,\Omega, \quad i_1 = 2\,A, \quad i_2 = 4\,A
\]

---

% SECTION 4: Nodal Analysis (Node Voltage Method)
\section*{4. Nodal Analysis (Node Voltage Method)}

\subsection*{4.1 Steps for Nodal Analysis}
\begin{enumerate}\itemsep0pt
    \item Identify essential nodes.
    \item Choose a reference node (ground).
    \item Write KCL equations at each essential node using node voltages.
    \item Solve the system of equations for unknown node voltages.
\end{enumerate}

\subsection*{4.2 Example of KCL at Node}
At node \( V_1 \):
\[
\frac{V_1 - V_{\text{source}}}{R_1} + \frac{V_1 - V_2}{R_2} = 0
\]
Solve the system of equations to find node voltages.

---

% SECTION 5: Mesh Analysis (Mesh Current Method)
\section*{5. Mesh Analysis (Mesh Current Method)}

\subsection*{5.1 Steps for Mesh Analysis}
\begin{enumerate}\itemsep0pt
    \item Identify meshes (loops without other loops inside).
    \item Assign mesh currents.
    \item Apply KVL in each mesh to write voltage equations.
    \item Solve the system of equations for mesh currents.
\end{enumerate}

\subsection*{5.2 Example of KVL in Mesh}
For a mesh with resistors \( R_1 \), \( R_2 \), and voltage source \( V_s \):
\[
i_a (R_1 + R_2) - i_b R_2 = V_s
\]
Solve for \( i_a \) and \( i_b \) in the system of equations.

---

% SECTION 6: Thevenin and Norton Equivalent Circuits
\section*{6. Thevenin and Norton Equivalent Circuits}

\subsection*{6.1 Thevenin's Theorem}
Any linear circuit can be reduced to a single voltage source \( V_{\text{Th}} \) in series with \( R_{\text{Th}} \).
\[
V_L=V_{Th}\frac{R_L}{R_{Th}+R_L}
\]

\subsection*{6.2 Norton's Theorem}
Any linear circuit can be reduced to a single current source \( I_N \) in parallel with \( R_N \).
\[
I_N = \frac{V_{\text{Th}}}{R_{\text{Th}}}, \quad R_N = R_{\text{Th}}
\]

---

% SECTION 7: Source Transformations
\section*{7. Source Transformations}

\subsection*{7.1 Voltage to Current Source Transformation}
A voltage source \( V_s \) in series with \( R \) can be transformed into a current source:
\[
I_s = \frac{V_s}{R}, \quad \text{in parallel with } R
\]

\subsection*{7.2 Current to Voltage Source Transformation}
A current source \( I_s \) in parallel with \( R \) can be transformed into a voltage source:
\[
V_s = I_s R, \quad \text{in series with } R
\]

---

% SECTION 8: Delta-Y ($\Delta$-Y) Conversion
\section*{8. Delta-Y ($\Delta$-Y) Conversion}

\subsection*{8.1 Delta to Y Conversion}
For a delta network with resistors \(R_a\), \(R_b\), and \(R_c\), the equivalent Y-resistances are:
\[
R_1 = \frac{R_b R_c}{R_a + R_b + R_c}, \quad 
R_2 = \frac{R_a R_c}{R_a + R_b + R_c}
\]
\[
R_3 = \frac{R_a R_b}{R_a + R_b + R_c}
\]
\textit{Example}: For a delta network with \(R_a = 10\,\Omega\), \(R_b = 20\,\Omega\), and \(R_c = 30\,\Omega\):
\[
R_1 = 12\,\Omega, \quad R_2 = 5\,\Omega, \quad R_3 = 3.33\,\Omega
\]

\subsection*{8.2 Y to Delta Conversion}
For a Y-network with resistors \(R_1\), \(R_2\), and \(R_3\), the equivalent delta-resistances are:
\[
R_a = \frac{R_1 R_2 + R_2 R_3 + R_3 R_1}{R_3}, \quad 
R_b = \frac{R_1 R_2 + R_2 R_3 + R_3 R_1}{R_1}
\]
\[
R_c = \frac{R_1 R_2 + R_2 R_3 + R_3 R_1}{R_2}
\]
\textit{Example}: For a Y-network with \(R_1 = 3\,\Omega\), \(R_2 = 4\,\Omega\), and \(R_3 = 5\,\Omega\):
\[
R_a = 9.33\,\Omega, \quad R_b = 7.67\,\Omega, \quad R_c = 6\,\Omega
\]

\end{multicols}
\end{document}
