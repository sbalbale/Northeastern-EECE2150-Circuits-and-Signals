\documentclass[10pt]{article}
\usepackage[margin=0.5in]{geometry}
\usepackage{amsmath, amssymb}
\usepackage{multicol}
\usepackage{graphicx}
\usepackage{circuitikz}

\setlength{\columnsep}{1.5cm} % Set the column separation to a larger value
\setlength{\voffset}{-0.25in}

\title{
    \raggedright
    \large EECE 2150: Quiz 2 Notesheet \hfill Sean Balbale \hfill September 25, 2024
    \vspace{-4em}
}
\date{}

\begin{document}
\maketitle

\begin{multicols}{2}

% SECTION 1: Resistors and Circuit Elements
\section*{1. Basic Circuit Elements}

\subsection*{1.1 Resistors}
\textbf{Resistor} resists the flow of charge. The resistance \(R\) is a function of length, area, and resistivity:
\[
R = \frac{\rho \ell}{A} \quad \text{or} \quad R = \frac{\ell}{\sigma A}
\]
\begin{itemize}\itemsep0pt
    \item \( \rho \): Resistivity
    \item \( \ell \): Length
    \item \( A \): Area
\end{itemize}
\textbf{Ohm's Law}: Voltage, current, and resistance are related:
\[
V = IR \quad \text{or} \quad I = \frac{V}{R}
\]
\textit{Example}: If \(V = 10\,V\) and \(R = 5\,\Omega\), then \(I = \frac{10\,V}{5\,\Omega} = 2\,A\).

\textbf{Power Dissipation}:
\[
P = IV = I^2 R = \frac{V^2}{R}
\]

\subsection*{1.2 Conductance}
Conductance is the reciprocal of resistance:
\[
G = \frac{1}{R} \quad \text{in Siemens (S)}
\]

\subsection*{1.3 Ideal Conductors}
\[
R = 0, \quad \sigma \to \infty
\]
No voltage drop across an ideal conductor.

% SECTION 2: Circuit Laws
\section*{2. Circuit Laws}

\subsection*{2.1 Kirchhoff's Current Law (KCL)}
The sum of currents entering a node equals the sum of currents leaving the node:
\[
\sum_{k=1}^{n} i_k = 0
\]
\textit{Example}: At a node, \(2\,A\), \(3\,A\), and \(5\,A\) enter. If \(i_x\) leaves, then:
\[
2 + 3 + 5 = i_x \quad \Rightarrow \quad i_x = 10\,A
\]

\subsection*{2.2 Kirchhoff's Voltage Law (KVL)}
The sum of voltage drops in a loop equals the sum of voltage rises:
\[
\sum_{k=1}^{n} v_k = 0
\]
\textit{Example}: In a loop with three voltage drops \(10\,V\), \(6\,V\), and \(4\,V\) and a supply voltage \(V_s = 24\,V\):
\[
10 + 6 + 4 + V_x = 24 \quad \Rightarrow \quad V_x = 4\,V
\]

% SECTION 3: Series and Parallel Resistors
\section*{3. Resistors in Series and Parallel}

\subsection*{3.1 Series Resistors}
Resistors in series share the same current:
\[
R_{\text{eq}} = R_1 + R_2 + \cdots + R_n
\]
\textbf{Voltage Division}: The voltage across \(R_k\) is:
\[
V_k = V_s \frac{R_k}{R_{\text{eq}}}
\]
\textit{Example}: For two resistors in series, \(R_1 = 10\,\Omega\) and \(R_2 = 5\,\Omega\), and the source voltage \(V_s = 30\,V\):
\[
V_1 = 30 \frac{10}{15} = 20\,V, \quad V_2 = 30 \frac{5}{15} = 10\,V
\]

\subsection*{3.2 Parallel Resistors}
Resistors in parallel share the same voltage:
\[
\frac{1}{R_{\text{eq}}} = \frac{1}{R_1} + \frac{1}{R_2} + \cdots + \frac{1}{R_n}
\]
\textbf{Current Division}:
\[
i_k = I_s \frac{R_{\text{eq}}}{R_k}
\]
\textit{Example}: For two resistors in parallel, \(R_1 = 6\,\Omega\) and \(R_2 = 3\,\Omega\), and a total current \(I_s = 6\,A\):
\[
\frac{1}{R_{\text{eq}}} = \frac{1}{6} + \frac{1}{3} = \frac{1}{2} \quad \Rightarrow \quad R_{\text{eq}} = 2\,\Omega
\]
\[
i_1 = 6\,A \times \frac{2}{6} = 2\,A, \quad i_2 = 6\,A \times \frac{2}{3} = 4\,A
\]

\subsection*{3.3 Series-Parallel Combination Example}
Find the equivalent resistance of the following circuit:
\[
R_1 = 4\,\Omega, \quad R_2 = 6\,\Omega \quad \text{(in parallel)}, \quad R_3 = 2\,\Omega \quad \text{(in series)}
\]
\[
\frac{1}{R_{\text{eq1}}} = \frac{1}{R_1} + \frac{1}{R_2} = \frac{1}{4} + \frac{1}{6} = \frac{5}{12} \quad \Rightarrow R_{\text{eq1}} = 2.4\,\Omega
\]
\[
R_{\text{total}} = R_{\text{eq1}} + R_3 = 2.4\,\Omega + 2\,\Omega = 4.4\,\Omega
\]

% SECTION 4: Dependent and Independent Sources
\section*{4. Sources}

\subsection*{4.1 Independent Sources}
\textbf{Ideal Voltage Source}: Constant voltage irrespective of current.\\
\textbf{Ideal Current Source}: Constant current irrespective of voltage.

\subsection*{4.2 Dependent Sources}
\textbf{Dependent Voltage Source}: Voltage depends on a current or voltage elsewhere.\\
\textbf{Dependent Current Source}: Current depends on a current or voltage elsewhere.

\subsection*{4.3 Example}
\begin{circuitikz}[scale=0.6]
\draw
(0,0) to [V=$V_x$] (0,3)
      to [R, l=$R_1$] (3,3)
      to [current source, l=$\alpha V_x$] (3,0)
      to [short] (0,0);
\end{circuitikz}

% SECTION 5: Equivalent Resistance and Delta-Wye Conversion
\section*{5. Equivalent Resistance and $\Delta$-Y Conversion}

\subsection*{5.1 Simplification of Resistive Networks}
To simplify circuits:
\begin{itemize}
    \item Combine series and parallel resistors.
    \item Reduce complex networks step-by-step.
\end{itemize}

\subsection*{5.2 Delta-Wye ($\Delta$-Y) Conversion}
\scriptsize % Make equations smaller
For resistors in a delta configuration, the equivalent Y-resistance is calculated as:

\[
R_1 = \frac{R_b R_c}{R_a + R_b + R_c}, \quad 
R_2 = \frac{R_a R_c}{R_a + R_b + R_c}, \quad 
R_3 = \frac{R_a R_b}{R_a + R_b + R_c}
\]

\textbf{Example}: For a delta network with \(R_a = 10\,\Omega\), \(R_b = 20\,\Omega\), and \(R_c = 30\,\Omega\):

\[
R_1 = \frac{20 \times 30}{10 + 20 + 30} = 12\,\Omega, \quad 
R_2 = \frac{10 \times 30}{60} = 5\,\Omega, \quad 
R_3 = \frac{10 \times 20}{60} = 3.33\,\Omega
\]
\normalsize % Return to normal size
\vspace{1em} % Add space between sections to prevent overlap

% SECTION 6: Passive Sign Convention
\section*{6. Passive Sign Convention}
In the passive sign convention:
\begin{itemize}
    \item If current enters through the positive terminal of an element, the element is absorbing power:
    \[
    P = VI
    \]
    \item If current enters through the negative terminal, the element is delivering power:
    \[
    P = -VI
    \]
\end{itemize}

\subsection*{6.1 Example: Power Calculation}
\textit{Example}: A \(12\,V\) source supplies a current of \(2\,A\) to a resistor. Power delivered by the source:
\[
P = -VI = -(12 \,V)(2\,A) = -24\,W
\]
This means the source is delivering \(24\,W\) of power.

% SECTION 7: Practice Problems
\section*{7. Practice Problems}

\subsection*{Problem 1: Ohm's Law}
A \(10\,\Omega\) resistor has a current of \(1.5\,A\). What is the voltage across it?

\subsection*{Solution:}
\[
V = IR = 1.5 \,A \times 10\, \Omega = 15\,V
\]

\subsection*{Problem 2: Equivalent Resistance}
Find the equivalent resistance of \(4\,\Omega\) and \(6\,\Omega\) in parallel.

\subsection*{Solution:}
\[
\frac{1}{R_{\text{eq}}} = \frac{1}{4} + \frac{1}{6} = \frac{5}{12} \quad \Rightarrow \quad R_{\text{eq}} = 2.4\,\Omega
\]

\end{multicols}
\end{document}
